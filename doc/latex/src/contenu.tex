\section{Spécifications}
L'objet principal de notre programme est de modéliser la chute d'un solide dans un
fluide, tout en prenant en compte son interaction avec un environnement solide (chocs).

\subsection{IHM - \emph{Interface Homme - Machine}}
Il est proposé à l'utilisateur de dessiner à la fois le décor et l'objet.
Les deux dessins sont séparés en deux fenêtres, munies chacunes de leur interface propre,
puisque les outils nécessaires au dessin de l'objet ne sont pas les mêmes que pour le dessin du décor.
La fenêtre principale est le lieu de la simulation, et affiche notamment les paramètres cinématiques
et spatiaux du mouvement (vitesses, position, angle).

\subsection{Problèmes physiques et méthodes de résolution}
\subsubsection{Forces}
Le programme doit prendre en compte les différentes forces appliquées à 
un instant $t$ sur le solide. Le mouvement de ce solide est donc déterminé
par les équations de la mécanique newtonienne.
On peut donc définir les différentes forces appliquées sur le solide :

\begin{itemize}
\item[$\bullet$] La gravité : \dotfill{} $\overrightarrow{P} = mg\, \overrightarrow{y_0}$ 
\item[$\bullet$] La poussée d'Archimède : \dotfill{} $\overrightarrow{F_{Ar}} = -\rho_{liquide}\cdot V_{deplace} \, \overrightarrow{y_0} $
\item[$\bullet$] Les frottements lors du contact: \dotfill{}$\overrightarrow{F_{fr}} = -f \,\overrightarrow{V}_{\text{Objet/Décor}}$
\end{itemize}

La majorité des forces sont calculées à l'aide d'algorithmes basiques de
calcul, prenant en compte des paramètres tels que le volume de l'objet,
sa vitesse ou son poids.
\subsubsection{\emph{PFD} - Principe Fondamental de la Dynamique }
Dans le cas d'un problème plan, notre problème s'exprime sous la forme :
\begin{equation}
\sum F_x = m \frac {dv_x}{dt} \: \text{et} \: \sum F_y = m\frac{dv_y}{dt} \: \text{et} \: \sum M_z = J \frac {d\omega}{dt}
\end{equation}
En supposant l'intervalle de temps $\Delta t$ du timer suffisament faible, on peut assimiler
$\frac {dv_x}{dt}$ à $\frac{\Delta v_x}{\Delta t}$,
$ \frac {dv_y}{dt} \approx \frac{\Delta v_y}{\Delta t}$ et
$ \frac{d\omega}{dt} \approx \frac{\Delta \omega}{\Delta t}$ .
\newline
Avec les mêmes approximations, on obtient :
$\frac {dx}{dt} = v_x \approx \frac{\Delta x}{\Delta t}$, 
$ \frac {dy}{dt} \approx \frac{\Delta y}{\Delta t}$ et
$ \frac{d\theta}{dt} \approx \frac{\Delta \theta}{\Delta t}$. \\
La vitesse initiale $\overrightarrow{v_0}$ est définie par l'utilisateur.
Ainsi, connaissant la vitesse $\overrightarrow{v(t)}$ à un instant $t$,
on peut calculer la vitesse $\overrightarrow{v(t+\Delta t)}$ à l'instant $t + \Delta t$.

\begin{equation}
v_{x}(t+\Delta t)= v_{x}(t) + m \Delta t \sum F_x \: \text{et}\:
v_{y}(t+\Delta t)= v_{y}(t) + m\Delta t\sum F_y \: \text{et}\:
\omega (t + \Delta t)= \omega(t) + J\Delta t \sum m_z 
\end{equation}
En prenant en compte les approximations précédentes, on obtient le système suivant :
\begin{equation}
x(t+\Delta t) = x(t) +\Delta t \cdot v_{x}(t) \: \text{et} \:
y(t+\Delta t) = x(t) +\Delta t \cdot v_{x}(t) \: \text{et} \:
\theta (t+\Delta t) = \theta (t) + \Delta t \cdot \omega (t) \: \text \:
\end{equation}

\newpage
\subsubsection{Collisions}
Notons $\overrightarrow{v_c}$ la vitesse du point de contact et $\overrightarrow{v_g}$
la vitesse, connue, du centre de gravité du solide.
Pour le calcul de la nouvelle vitesse du solide, nous supposons que le contact
entre le solide et le décor lors du choc est ponctuel. 
On se place dans la base de contact définie par le schéma suivant :

\begin{center}
\begin{figure}[h]
\begin{center}
\scalebox{0.2}[0.2]{\includegraphics*{../images/collision.png}}
\end{center}
\caption{Base locale du contact}
\end{figure}
\end{center}
Le contact étant ponctuel, on suppose que les forces appliquées au point
de contact C sont similaires à celles appliquées dans le modèle de choc d'une particule, on a donc :
\begin{equation}
v_{C,x,initial} = v_{C,x,final} \: \text{et} \: v_{C,y,initial} = -v_{C,y,final}
\end{equation}
Or, en notant $x_G$ et $y_G$ les coordonnées du centre de gravité dans 
la base de contact :
\begin{equation}
v_{C,x}=v_{G,x}+y_G \cdot \omega \: \text{et} \: v_{C,y} = v_{g,y} -x_G\cdot \omega
\end{equation}
On suppose dans un premier temps que l'énergie cinétique est conservée :
\begin{equation}
J\omega^2_{initial} + m(v^2_{g,x,initial} + v^2_{g,y,initial} )
=
J\omega^2_{final} + m(v^2_{g,x,final} + v^2_{g,y,final} )
\end{equation}
On obtient ainsi un système de 3 équations qui nous donne les équations de
mouvement après collision.
$(v_{x0} \: , \: v_{x1})$ désigne $(v_{G,x,initial} \: , \: v_{G,y,initial})$,
de même pour $y$ et $\omega$ .\\
On obtient ainsi les solutions suivantes :
\begin{equation}
\omega_1 = \frac{-m(x^2_G+v_{x0}y_G+y^2_G\omega_0 + v_{y0}) +\sqrt{\Delta}}{my^2_G + mx^2_G +J}
\end{equation}

\begin{eqnarray*}
	\Delta & = & \omega^2 J^2 -2Jmv_{x0}y_G\omega_0 + 2Jmv_{y0}x_G \omega_0  + 2m^2v_{x0}y_Gv_{y0}x_G - 4m^2v_{x0}y_Gx^2_G\omega_0 \\
		& & +\ 4m^2y^2_g\omega_0 v_{y0} x_G- 4m^2y^2_g\omega^2_0x^2_g + m^2v^2_{x0}y^2_G + m^2v^2_{y0}x^2_G
\end{eqnarray*}

On obtient ainsi :
\begin{equation}
v_{x1}=v_{x0} +y_G(\omega_0 - \omega_1) \: \ \text{et} \ \:  v_{y1}=v_{y0} +x_G(\omega_0 - \omega_1)
\end{equation} 

\newpage

\section{Réalisation}
\subsection{Interface}
La fenêtre principale affiche un menu utilisateur, une fenêtre de dessin,
et un espace d'affichage des principaux paramètres physiques.
Outre ces fonctionnalités, c'est également à elle d'effectuer l'animation
de l'image, en calculant régulièrement (présence d'un Timer) la position
du solide ainsi que sa vitesse et son interaction avec le décor.
Les autres fenêtres ont pour mission de gérer le dessin de l'objet et son remplissage,
le dessin du décor et la gestion des matériaux.


\subsection{Gestion des chocs}
La partie la plus compliquée dans la gestion des chocs reste la détection des chocs
et le calcul de la tangente locale au décor.

A chaque itération du timer, une procédure est lancée après le calcul de 
la nouvelle position de l'objet pour détecter une collision potentielle.
On parcourt la matrice du solide et l'extrait correspondant de la matrice
du décor pour chercher d'éventuelles superpositions de pixels non blancs.
Les dimensions horizontales et verticales de l'objet ont été au préalable
calculées pour optimiser ce parcours.

Il s'agit ensuite de déterminer la zone restreinte du contact pour la réduire ensuite
à un unique pixel qui servira dans les calculs dynamiques.
Pour ce faire, une boucle fait reculer (sans l'afficher) le solide dans 
la direction opposée à la vitesse, pixel par pixel,
jusqu'à ce qu'il n'y ait plus de collision.

La procédure fait aussi tourner l'objet petit à petit pour retrouver
son angle de rotation avant le choc.
L'ensemble des pixels colorés du décor superposés aux points du solide
sont stockés dans un tableau. Lorsque la boucle se termine, ce tableau
contient les points de la surface du décor qui délimitent la zone de contact.
On prend ensuite un point au milieu de cette zone que l'on définit comme pixel de collision.\\

\begin{center}
\begin{figure}[h]
\begin{center}
\scalebox{0.3}[0.3]{\includegraphics*{../images/tangente.png}}
\end{center}
\caption{Schématisation du calcul de la tangente}
\end{figure}
\end{center}

Le calcul des tangentes se fait à partir du pixel de contact et des deux pixels
constituant la surface les plus proches du pixel de contact.

Pour déterminer ces deux pixels, la procédure teste chaque pixel adjacent
au pixel de contact et s'arrête sur les premiers pixels pleins trouvés.

Les deux premiers pixels testés sont déterminés par la vitesse à laquelle
l'objet arrive vers la surface (cf schéma ci-dessus).

Si aucun de ces pixels n'est plein (ce qui en pratique devrait toujours être le cas),
une procédure teste tour à tour les pixels en tournant autour du point de contact,
dans le sens horaire pour un point et dans le sens trigonométrique pour l'autre.

Une fois les deux points déterminés, le vecteur tangent est calculé en
faisant la moyenne du vecteur reliant le point de contact et le premier point
et du vecteur reliant le point de contact et le deuxième point.



\subsection{Remplissage de l'objet}
Un des problèmes majeurs qui nous fut posé durant ce projet fut le remplissage
du solide afin de pouvoir déterminer sa masse, son volume et son centre d'inertie.
Tout d'abord, il fallut s'assurer que le dessin de l'utilisateur était bien connexe.
Pour remédier à ce problème, nous traçons une droite entre le point de départ
$(X_e,Y_e)$ et le point de sortie $(X_s,Y_s)$, correspondant respectivement au clic
gauche de la souris et au lacher de ce bouton (maintenu pendant le dessin).
Nous avons choisi la méthode du ScanLine-FloodFill pour remplir le solide dessiné.
Cette méthode utilise une graine (la \emph{seed}) placée sur un point quelquonque
de notre solide. À partir de ce point, on remplit une ligne verticale contenue dans l'objet
et analyse les pixels mitoyens pour déterminer
quels sont ceux à l'intérieur du solide, et les autres. Par récursivité, les pixels
blancs à gauche et à droite de la ligne tracée deviennent des graines et l'algorithme
est relancé, jusqu'à remplissage complet de l'objet connexe.
Après plusieurs essais infructueux avec un algorithme classique, nous avons
adapté un algorithme en C trouvé sur Internet, qui analyse ligne par ligne l'objet,
permettant ainsi un gain considérable en efficacité. 
%
%\begin{algorithm}[h]
%\caption{Pseudo algorithme de remplissage par ligne}
%\begin{algorithmic}
%	\STATE Prout;
%	\PROCEDURE {RemplirLigne} {$(aOldColor,aNewColor, X_{seed}, Y_{seed})$}
%		\WHILE{$couleur(Y+1) = aOldColor$ and $couleur(Y+1) \neq couleur_{\text{frontière}}$}
%			\STATE $couleur(Y+1) \gets aNewColor$
%		\ENDWHILE
%		\IF{$couleur_{voisin\_ gauche} = aOldColor$}
%			\STATE remplir\_ Ligne(X-1,Y)
%		\ENDIF
%		\IF{$couleur_{voisin\_ droit} = aOldColor$}
%			\STATE remplir\_ Ligne(X+1,Y)
%		\ENDIF
%	\ENDPROCEDURE
%\end{algorithmic}
%\end{algorithm}
%\begin{algorithmic}[1]
%\repeat
%\Comment{forever}
%\State this\Until{you die.}
%\end{algorithmic}

\newpage

\section{Bugs - Améliorations}
Il subsiste de nombreux bugs et améliorations possibles de notre projet.
En particulier, de nombreuses améliorations que nous pensions apporter
à notre projet n'ont pas été implémentées par faute de temps, certains 
problèmes ayant été plus chronophages que prévu.
\subsection{Améliorations}
\begin{itemize}
\item[$\bullet$] Différents fluides (air, eau, vide \dots)
\item[$\bullet$] Variation aléatoire et incidence du vent 
\item[$\bullet$] Amélioration du modèle physique (élasticité \dots )
\item[$\bullet$] Modification du décor et de l'objet en temps réel par l'utilisateur
\item[$\bullet$] Tracé de la trajectoire de l'objet
\item[$\bullet$] Force de frottement prenant en compte la forme de l'objet
\item[$\bullet$] Roulement ou arrêt de l'objet possible
\item[$\bullet$] Possibilité de modification des paramètres physiques (prévu mais pas implémenté)
\end{itemize}
\subsection{Bugs}
\begin{itemize}
\item[$\bullet$] Lenteur de l'exécution sous Linux (pas de problèmes sous Windows)
\item[$\bullet$] Le décor ne se remet pas à jour si on le redessine
\item[$\bullet$] Calcul des tangentes peu précis (sinon il serait beaucoup trop lent)
\item[$\bullet$] Mauvais fonctionnement sur certains ordinateurs (64 bits?)
\end{itemize}
