\section{Spécifications}
L'objet principal de notre programme est de modéliser la chute d'un solide dans un
fluide quelquonque, en interagissant avec son environnement.
\subsection{IHM - \emph{Interface Homme - Machine}}
Il est proposé à l'utilisateur de dessiner à la fois le décor et l'objet.
Les deux dessins sont séparés en deux fenêtres, munie chacunes de leur interface propre,
puisque les outils nécessaires au dessin du décor ne sont pas les mêmes que pour le dessin de l'objet.
L'utilisateur peut également ouvrir une fenêtre dédiée aux paramètres physiques, afin d'influer
sur la valeur de la gravité notamment.
\subsection{Problèmes physiques et méthodes de résolution}
Le programme doit prendre en compte les différentes forces appliquées à 
un instant $t$ sur le solide. Le mouvement de ce solide est donc déterminé
par les équations de la mécanique newtonienne.
On peut donc définir les différentes forces appliquées sur le solide :
\begin{itemize}
\item[$\bullet$] La gravité : \dotfill{} $\overrightarrow{P} = -m*g*\overrightarrow{y_0}$ 
\item[$\bullet$] La poussée d'Archimède : \dotfill{} $\overrightarrow{F_{Ar}} = -\rho_{liquide} * V_{deplace} *\overrightarrow{y_0} $
\item[$\bullet$] Les frottements lors du contact: \dotfill{}$\overrightarrow{F_{fr}} = -f*\overrightarrow{V}_{\text{Objet/Décor}}$
\item[$\bullet$] La force lors du choc : \dotfill{} $ \|\overrightarrow{V_r} \|= e * \| \overrightarrow{V_i} \|$ avec $ e \in [ 0 , 1 ] $\\
\end{itemize}

La majorité des forces sont résolues à l'aide d'algorithmes basiques de
calcul, prenant en compte des paramètres telles que le volume de l'objet,
sa vitesse ou son poids. Les deux forces nécessitant de gros algorithmes
de traitement sont les forces de frottement et le contact entre le solide
et le décor.
Pour la gestion du contact entre les deux solides, nous avons privilégiés
une approche 

\section{Réalisation}
\subsection{Interface}
\subsection{Gestion des chocs}
Calcul tangentes - Inversion force avec coeff
\subsection{Remplissage de l'objet}
FloodFill Scan Line


\section{Bugs - Améliorations}
Il subsiste de nombreux bugs et améliorations possibles de notre projet.
En particulier, de nombreuses améliorations que nous pensions apporter
à notre projet n'ont pas été implémentées par faute de temps, certains 
problèmes ayant été plus chronophages que prévu.
\subsection{Améliorations}
\begin{itemize}
\item[$\bullet$] Différents fluides (air, eau, vide \dots)
\item[$\bullet$] Variation aléatoire et incidence du vent 
\item[$\bullet$] Amélioration du modèle physique (élasticité \dots )
\item[$\bullet$] Modification du décor et de l'objet en temps réel par l'utilisateur
\item[$\bullet$] Tracé de la trajectoire de l'objet
\end{itemize}
\subsection{Bugs}
\begin{itemize}
\item[$\bullet$] Lenteur de l'exécution sous Linux (pas de problèmes sous Windows)
\item[$\bullet$] Le décor ne se remet pas à jour si on le redessine
\item[$\bullet$] Calcul des tangentes peu précis (sinon il serait beaucoup trop lent)
\end{itemize}
